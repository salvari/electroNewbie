\section{¿Qué vamos a hacer? ¿Qué se describe en este
documento?}\label{quuxe9-vamos-a-hacer-quuxe9-se-describe-en-este-documento}

Apuntes de electrónica por supuesto.

\section{LEDS}\label{leds}

\begin{quote}
Wikipedia ``Un led (del acrónimo inglés LED, light-emitting diode:
`diodo emisor de luz'; el plural aceptado por la RAE es ledes ) es un
componente optoelectrónico pasivo y, más concretamente, un diodo que
emite luz.''
\end{quote}

\subsection{Calculo de resistencias para
leds}\label{calculo-de-resistencias-para-leds}

\textbf{NO ESTÁ TERMINADO}

Un led y en general cualquier diodo es una unión de dos elementos
semiconductores. Este tipo de uniones tienen una caída de voltaje de
valor fijo (al menos en el rango de funcionamiento normal).

\begin{longtable}[c]{@{}lc@{}}
\caption{Caída de tensión típica en leds}\tabularnewline
\toprule
\begin{minipage}[b]{0.31\columnwidth}\raggedright\strut
Color del led
\strut\end{minipage} &
\begin{minipage}[b]{0.12\columnwidth}\centering\strut
Tensión umbral
\strut\end{minipage}\tabularnewline
\midrule
\endfirsthead
\toprule
\begin{minipage}[b]{0.31\columnwidth}\raggedright\strut
Color del led
\strut\end{minipage} &
\begin{minipage}[b]{0.12\columnwidth}\centering\strut
Tensión umbral
\strut\end{minipage}\tabularnewline
\midrule
\endhead
\begin{minipage}[t]{0.31\columnwidth}\raggedright\strut
Rojo
\strut\end{minipage} &
\begin{minipage}[t]{0.12\columnwidth}\centering\strut
1,6 V
\strut\end{minipage}\tabularnewline
\begin{minipage}[t]{0.31\columnwidth}\raggedright\strut
Rojo alta luminosidad
\strut\end{minipage} &
\begin{minipage}[t]{0.12\columnwidth}\centering\strut
1,9 V
\strut\end{minipage}\tabularnewline
\begin{minipage}[t]{0.31\columnwidth}\raggedright\strut
Amarillo
\strut\end{minipage} &
\begin{minipage}[t]{0.12\columnwidth}\centering\strut
1,7 2 V
\strut\end{minipage}\tabularnewline
\begin{minipage}[t]{0.31\columnwidth}\raggedright\strut
Verde
\strut\end{minipage} &
\begin{minipage}[t]{0.12\columnwidth}\centering\strut
2,4 V
\strut\end{minipage}\tabularnewline
\begin{minipage}[t]{0.31\columnwidth}\raggedright\strut
Naranja
\strut\end{minipage} &
\begin{minipage}[t]{0.12\columnwidth}\centering\strut
2,4 V
\strut\end{minipage}\tabularnewline
\begin{minipage}[t]{0.31\columnwidth}\raggedright\strut
Blanco Brillante
\strut\end{minipage} &
\begin{minipage}[t]{0.12\columnwidth}\centering\strut
3,4 V
\strut\end{minipage}\tabularnewline
\begin{minipage}[t]{0.31\columnwidth}\raggedright\strut
Azul
\strut\end{minipage} &
\begin{minipage}[t]{0.12\columnwidth}\centering\strut
3,4 V
\strut\end{minipage}\tabularnewline
\begin{minipage}[t]{0.31\columnwidth}\raggedright\strut
Azul 430nm
\strut\end{minipage} &
\begin{minipage}[t]{0.12\columnwidth}\centering\strut
4,6 V
\strut\end{minipage}\tabularnewline
\bottomrule
\end{longtable}

La corriente que necesita el led también depende del tipo de led, pero
en general están entre 10mA y 30mA. Con corrientes bajas obtendremos
poco brillo y el led durará mucho. Con corrientes altas obtendremos más
brillo y una vida útil más corta. Un valor de 20mA es un buen compromiso
entre ambos extremos.

\section{META}\label{meta}

De momento nos proponemos usar pandoc para generar los documentos, en
esta sección describiremos detalladamente como se generan los distintos
documentos de salida usando pandoc.
